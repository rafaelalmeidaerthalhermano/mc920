\documentclass[10pt,a4paper]{article}
\usepackage[utf8]{inputenc}
\usepackage{ae}
\usepackage[brazil]{babel}
\usepackage[vmargin=2cm,hmargin=2cm,columnsep=0.75cm]{geometry}
\usepackage{float,nonfloat}
\usepackage{graphicx,color}
\usepackage{subcaption}
\usepackage{amsmath}
\usepackage{verbatim}

\makeatletter
\let\@institution\empty
\def\institution#1{\def\@institution{#1}}
\renewcommand{\maketitle}{
    \begin{center}
        {\Large\bfseries\@title\par\medskip}
        {\large
            \begin{tabular}[t]{c}%
                \@author
        \end{tabular}\par\medskip}
        {\itshape\@institution\par}
        {\itshape\@date\par}
\end{center}}
\makeatother

\newcommand{\pixel}{\textit{pixel} }
\newcommand{\pixels}{\textit{pixels} }
\newcommand{\kernel}{\textit{kernel} }
\newcommand{\kernels}{\textit{kernels} }

\begin{document}
% ============================================================================

\title{MC920: Introdução ao Processamento de Imagem Digital\\Questões G2 $\rightarrow$ G4}
\author{
    \begin{minipage}{6cm}
        \centering
        Martin Ichilevici de Oliveira\\
        RA 118077
    \end{minipage}
    \and
    \begin{minipage}{6cm}
        \centering
        Rafael Almeida Erthal Hermano\\
        RA 121286
    \end{minipage}
}
\institution{Instituto de Computação, Universidade Estadual de Campinas}
\date{\today}

\newcommand{\itemm}[1]{\item \textbf{#1}}

\maketitle

% ============================================================================

\section{Difusão anisotrópica}

\begin{enumerate}
\itemm{Qual a relação entre os filtros passa-altas e passa-baixas com o contraste das imagens?}

Filtros passa-altas, como o próprio nome já diz, permite a passagem de altas frequências, que são características de imagens bem definidas. Assim, este tipo de filtro aumenta o contraste da imagem. Já os filtros passa-baixas, tais como o \textit{Gaussian Blur}, tem por objetivo remover ruídos, através do borramento da imagem. Assim, estes filtros diminuem o contraste de uma imagem.

\itemm{Derivadas parciais compõem o gradiente de uma imagem e são lineares, porém não são isotrópicas. Por quê? E quanto à magnitude do gradiente?}
\itemm{Como seriam as máscaras dos filtros de Prewitt e Sobel para a detecção de bordas diagonais?}
As máscaras de Prewitt seriam dadas por

\[P_x = \left[\begin{array}{ccc}
  -1 & -1 & 0 \\
  -1 &  0 & 1 \\
  0  &  1 & 1
\end{array}\right]
\qquad
P_y = \left[\begin{array}{ccc}
  0 & 1 & 1 \\
  -1 &  0 & 1 \\
  -1  & -1 & 0
\end{array}\right]
\]

Já a máscara de Sobel seria dada por

\[S_x = \left[\begin{array}{ccc}
  -2 & -1 & 0 \\
  -1 &  0 & 1 \\
  0  &  1 & 2
\end{array}\right]
\qquad
S_y = \left[\begin{array}{ccc}
  0 & 1 & 2 \\
  -1 &  0 & 1 \\
  -2  & -1 & 0
\end{array}\right]
\]
\itemm{Seria possível detectar o gradiente de uma imagem através da convolução de um filtro de ordem estatística (e.g. filtros de máximo e mínimo)? Como?}

  Não. Um filtro de ordem estatística desconsidera o arranjo dos \pixels em uma imagem, e leva em conta apenas seus valores. Considere duas imagens $3\times3$ com o mesmo histograma (isto é, com a mesma quantidade de \pixels em cada nível de intensidade), mas cuja distribuição de \pixels fosse diferente. Se fosse aplicada um filtro de ordem estatística, estes produziriam o mesmo resultado. Contudo, é claro que seus gradientes devem ser diferentes. Afinal, o gradiente, por sua própria natureza, considera o conjunto de pontos ao redor do centro, o que não é feito em filtros de ordem estatística -- apenas um deles é escolhido (tal como o mínimo, o máximo ou a mediana).
\end{enumerate}

\begin{thebibliography}{99}
    \bibitem{livro2} CRISTOBAL, G.; SCHELKENS, P.; THIENPONT, H. \textbf{Optical and Digital Image Processing: Fundamentals and Applications}. 1. ed. Wiley-VCH, 2011.
    \bibitem{high-pass} \texttt{http://www.digitalphotopro.com/software/image-processing/the-high-pass-filter.html}
\end{thebibliography}

\end{document}
