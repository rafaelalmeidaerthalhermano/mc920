\documentclass[10pt,a4paper]{article}
\usepackage[utf8]{inputenc}
\usepackage{ae}
\usepackage[brazil]{babel}
\usepackage[vmargin=2cm,hmargin=2cm,columnsep=0.75cm]{geometry}
\usepackage{float,nonfloat}
\usepackage{graphicx,color}
\usepackage{subcaption}
\usepackage{amsmath}

\makeatletter
\let\@institution\empty
\def\institution#1{\def\@institution{#1}}
\renewcommand{\maketitle}{
    \begin{center}
        {\Large\bfseries\@title\par\medskip}
        {\large
            \begin{tabular}[t]{c}%
                \@author
        \end{tabular}\par\medskip}
        {\itshape\@institution\par}
        {\itshape\@date\par}
\end{center}}
\makeatother

\newcommand{\pixel}{\textit{pixel} }
\newcommand{\pixels}{\textit{pixels} }

\begin{document}
% ============================================================================

\title{MC920: Introdução ao Processamento de Imagem Digital\\Tarefa 5}
\author{
    \begin{minipage}{6cm}
        \centering
        Martin Ichilevici de Oliveira\\
        RA 118077
    \end{minipage}
    \and
    \begin{minipage}{6cm}
        \centering
        Rafael Almeida Erthal Hermano\\
        RA 121286
    \end{minipage}
}
\institution{Instituto de Computação, Universidade Estadual de Campinas}
\date{\today}

\maketitle

% ============================================================================

\section{Ruído}
\subsection{Bipolar}
A função de distribuição de probabilidade de um ruído bipolar é dada por:

\begin{equation}
p(z) = \left\{
    \begin{array}{l}
        P_a \text{ se } z = a \\
        P_b \text{ se } z = b \\
        0 \text{ caso contrário}
    \end{array}\right.
\end{equation}

Onde $P_a$ é o clareamento do pixel e $P_b$ é o escurecimento de um pixel.

\subsubsection{Salt and pepper}
Se $P_a$ e $P_b$ forem similares, temos um ruído salt and pepper.

\subsubsection{Unipolar}
Se $P_a \cdot P_b = 0$ temos um ruído unipolar ja que apenas uma das faixas será modificada.

\subsection{Erlang}
A função de ditribuição de probabilidade Erlang foi originalmente desenvolvida para modelar trafego de ligações telefonicas.

\begin{equation}
p(z) = \left\{
    \begin{array}{l}
        \frac{a^{b}\cdot z^{b-1}}{(b - 1)!} \cdot e^{-a z} \text{ se } z \ge 0 \\
        0 \text{ se } z < 0
    \end{array}\right.
\end{equation}

Onde $b$ é a forma e $a$ é a taxa de crescimento.

\subsubsection{Exponencial}
Se $b = 1$ temos um ruido exponencial

\begin{equation}
p(z) = \left\{
    \begin{array}{l}
        a \cdot e^{-a z} \text{ se } z \ge 0 \\
        0 \text{ se } z < 0
    \end{array}\right.
\end{equation}

\subsection{Rayleigh}
Como o função de distribuição de probabilidade de Rayleigh possui uma assimetria entre os lados direito e esquerdo, ela pode ser utilizado para aproximar histogramas assimétricos.

\begin{equation}
p(z) = \left\{
    \begin{array}{l}
        \frac{1}{b} \cdot (z - a) \cdot e^{\frac{-(z-a)^2}{b}} \text{ se } z \ge 0 \\
        0 \text{ se } z < 0
    \end{array}\right.
\end{equation}

\subsection{Uniforme}
A função distribuição de probabilidade uniforme eleva os níveis de cinza na faixa de $a$ até $b$

\begin{equation}
p(z) = \left\{
    \begin{array}{l}
        \frac{1}{b - a} \text{ se } a \le z \le b \\
        0 \text{ se } z < 0
    \end{array}\right.
\end{equation}

\subsection{Branco}
Martin

\subsection{Gaussiano}
Carioca

\section{Filtro da média}
Martin

\section{Filtro Gaussiano}
Carioca

\section{Mediana}
Martin

\begin{thebibliography}{99}
    \bibitem{livro} GONZALEZ, Rafael C.; WOODS, Richard E.. \textbf{Digital Image Processing}. 3. ed. Upper Saddle River, NJ, EUA: Prentice-hall, 2006.
    \bibitem{white_noise} http\:\/\/en.wikipedia.org\/wiki\/White\_noise
\end{thebibliography}

\end{document}
